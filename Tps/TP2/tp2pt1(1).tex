\documentclass[11pt]{article}

    \usepackage[breakable]{tcolorbox}
    \usepackage{parskip} % Stop auto-indenting (to mimic markdown behaviour)
    

    % Basic figure setup, for now with no caption control since it's done
    % automatically by Pandoc (which extracts ![](path) syntax from Markdown).
    \usepackage{graphicx}
    % Keep aspect ratio if custom image width or height is specified
    \setkeys{Gin}{keepaspectratio}
    % Maintain compatibility with old templates. Remove in nbconvert 6.0
    \let\Oldincludegraphics\includegraphics
    % Ensure that by default, figures have no caption (until we provide a
    % proper Figure object with a Caption API and a way to capture that
    % in the conversion process - todo).
    \usepackage{caption}
    \DeclareCaptionFormat{nocaption}{}
    \captionsetup{format=nocaption,aboveskip=0pt,belowskip=0pt}

    \usepackage{float}
    \floatplacement{figure}{H} % forces figures to be placed at the correct location
    \usepackage{xcolor} % Allow colors to be defined
    \usepackage{enumerate} % Needed for markdown enumerations to work
    \usepackage{geometry} % Used to adjust the document margins
    \usepackage{amsmath} % Equations
    \usepackage{amssymb} % Equations
    \usepackage{textcomp} % defines textquotesingle
    % Hack from http://tex.stackexchange.com/a/47451/13684:
    \AtBeginDocument{%
        \def\PYZsq{\textquotesingle}% Upright quotes in Pygmentized code
    }
    \usepackage{upquote} % Upright quotes for verbatim code
    \usepackage{eurosym} % defines \euro

    \usepackage{iftex}
    \ifPDFTeX
        \usepackage[T1]{fontenc}
        \IfFileExists{alphabeta.sty}{
              \usepackage{alphabeta}
          }{
              \usepackage[mathletters]{ucs}
              \usepackage[utf8x]{inputenc}
          }
    \else
        \usepackage{fontspec}
        \usepackage{unicode-math}
    \fi

    \usepackage{fancyvrb} % verbatim replacement that allows latex
    \usepackage{grffile} % extends the file name processing of package graphics
                         % to support a larger range
    \makeatletter % fix for old versions of grffile with XeLaTeX
    \@ifpackagelater{grffile}{2019/11/01}
    {
      % Do nothing on new versions
    }
    {
      \def\Gread@@xetex#1{%
        \IfFileExists{"\Gin@base".bb}%
        {\Gread@eps{\Gin@base.bb}}%
        {\Gread@@xetex@aux#1}%
      }
    }
    \makeatother
    \usepackage[Export]{adjustbox} % Used to constrain images to a maximum size
    \adjustboxset{max size={0.9\linewidth}{0.9\paperheight}}

    % The hyperref package gives us a pdf with properly built
    % internal navigation ('pdf bookmarks' for the table of contents,
    % internal cross-reference links, web links for URLs, etc.)
    \usepackage{hyperref}
    % The default LaTeX title has an obnoxious amount of whitespace. By default,
    % titling removes some of it. It also provides customization options.
    \usepackage{titling}
    \usepackage{longtable} % longtable support required by pandoc >1.10
    \usepackage{booktabs}  % table support for pandoc > 1.12.2
    \usepackage{array}     % table support for pandoc >= 2.11.3
    \usepackage{calc}      % table minipage width calculation for pandoc >= 2.11.1
    \usepackage[inline]{enumitem} % IRkernel/repr support (it uses the enumerate* environment)
    \usepackage[normalem]{ulem} % ulem is needed to support strikethroughs (\sout)
                                % normalem makes italics be italics, not underlines
    \usepackage{soul}      % strikethrough (\st) support for pandoc >= 3.0.0
    \usepackage{mathrsfs}
    

    
    % Colors for the hyperref package
    \definecolor{urlcolor}{rgb}{0,.145,.698}
    \definecolor{linkcolor}{rgb}{.71,0.21,0.01}
    \definecolor{citecolor}{rgb}{.12,.54,.11}

    % ANSI colors
    \definecolor{ansi-black}{HTML}{3E424D}
    \definecolor{ansi-black-intense}{HTML}{282C36}
    \definecolor{ansi-red}{HTML}{E75C58}
    \definecolor{ansi-red-intense}{HTML}{B22B31}
    \definecolor{ansi-green}{HTML}{00A250}
    \definecolor{ansi-green-intense}{HTML}{007427}
    \definecolor{ansi-yellow}{HTML}{DDB62B}
    \definecolor{ansi-yellow-intense}{HTML}{B27D12}
    \definecolor{ansi-blue}{HTML}{208FFB}
    \definecolor{ansi-blue-intense}{HTML}{0065CA}
    \definecolor{ansi-magenta}{HTML}{D160C4}
    \definecolor{ansi-magenta-intense}{HTML}{A03196}
    \definecolor{ansi-cyan}{HTML}{60C6C8}
    \definecolor{ansi-cyan-intense}{HTML}{258F8F}
    \definecolor{ansi-white}{HTML}{C5C1B4}
    \definecolor{ansi-white-intense}{HTML}{A1A6B2}
    \definecolor{ansi-default-inverse-fg}{HTML}{FFFFFF}
    \definecolor{ansi-default-inverse-bg}{HTML}{000000}

    % common color for the border for error outputs.
    \definecolor{outerrorbackground}{HTML}{FFDFDF}

    % commands and environments needed by pandoc snippets
    % extracted from the output of `pandoc -s`
    \providecommand{\tightlist}{%
      \setlength{\itemsep}{0pt}\setlength{\parskip}{0pt}}
    \DefineVerbatimEnvironment{Highlighting}{Verbatim}{commandchars=\\\{\}}
    % Add ',fontsize=\small' for more characters per line
    \newenvironment{Shaded}{}{}
    \newcommand{\KeywordTok}[1]{\textcolor[rgb]{0.00,0.44,0.13}{\textbf{{#1}}}}
    \newcommand{\DataTypeTok}[1]{\textcolor[rgb]{0.56,0.13,0.00}{{#1}}}
    \newcommand{\DecValTok}[1]{\textcolor[rgb]{0.25,0.63,0.44}{{#1}}}
    \newcommand{\BaseNTok}[1]{\textcolor[rgb]{0.25,0.63,0.44}{{#1}}}
    \newcommand{\FloatTok}[1]{\textcolor[rgb]{0.25,0.63,0.44}{{#1}}}
    \newcommand{\CharTok}[1]{\textcolor[rgb]{0.25,0.44,0.63}{{#1}}}
    \newcommand{\StringTok}[1]{\textcolor[rgb]{0.25,0.44,0.63}{{#1}}}
    \newcommand{\CommentTok}[1]{\textcolor[rgb]{0.38,0.63,0.69}{\textit{{#1}}}}
    \newcommand{\OtherTok}[1]{\textcolor[rgb]{0.00,0.44,0.13}{{#1}}}
    \newcommand{\AlertTok}[1]{\textcolor[rgb]{1.00,0.00,0.00}{\textbf{{#1}}}}
    \newcommand{\FunctionTok}[1]{\textcolor[rgb]{0.02,0.16,0.49}{{#1}}}
    \newcommand{\RegionMarkerTok}[1]{{#1}}
    \newcommand{\ErrorTok}[1]{\textcolor[rgb]{1.00,0.00,0.00}{\textbf{{#1}}}}
    \newcommand{\NormalTok}[1]{{#1}}

    % Additional commands for more recent versions of Pandoc
    \newcommand{\ConstantTok}[1]{\textcolor[rgb]{0.53,0.00,0.00}{{#1}}}
    \newcommand{\SpecialCharTok}[1]{\textcolor[rgb]{0.25,0.44,0.63}{{#1}}}
    \newcommand{\VerbatimStringTok}[1]{\textcolor[rgb]{0.25,0.44,0.63}{{#1}}}
    \newcommand{\SpecialStringTok}[1]{\textcolor[rgb]{0.73,0.40,0.53}{{#1}}}
    \newcommand{\ImportTok}[1]{{#1}}
    \newcommand{\DocumentationTok}[1]{\textcolor[rgb]{0.73,0.13,0.13}{\textit{{#1}}}}
    \newcommand{\AnnotationTok}[1]{\textcolor[rgb]{0.38,0.63,0.69}{\textbf{\textit{{#1}}}}}
    \newcommand{\CommentVarTok}[1]{\textcolor[rgb]{0.38,0.63,0.69}{\textbf{\textit{{#1}}}}}
    \newcommand{\VariableTok}[1]{\textcolor[rgb]{0.10,0.09,0.49}{{#1}}}
    \newcommand{\ControlFlowTok}[1]{\textcolor[rgb]{0.00,0.44,0.13}{\textbf{{#1}}}}
    \newcommand{\OperatorTok}[1]{\textcolor[rgb]{0.40,0.40,0.40}{{#1}}}
    \newcommand{\BuiltInTok}[1]{{#1}}
    \newcommand{\ExtensionTok}[1]{{#1}}
    \newcommand{\PreprocessorTok}[1]{\textcolor[rgb]{0.74,0.48,0.00}{{#1}}}
    \newcommand{\AttributeTok}[1]{\textcolor[rgb]{0.49,0.56,0.16}{{#1}}}
    \newcommand{\InformationTok}[1]{\textcolor[rgb]{0.38,0.63,0.69}{\textbf{\textit{{#1}}}}}
    \newcommand{\WarningTok}[1]{\textcolor[rgb]{0.38,0.63,0.69}{\textbf{\textit{{#1}}}}}


    % Define a nice break command that doesn't care if a line doesn't already
    % exist.
    \def\br{\hspace*{\fill} \\* }
    % Math Jax compatibility definitions
    \def\gt{>}
    \def\lt{<}
    \let\Oldtex\TeX
    \let\Oldlatex\LaTeX
    \renewcommand{\TeX}{\textrm{\Oldtex}}
    \renewcommand{\LaTeX}{\textrm{\Oldlatex}}
    % Document parameters
    % Document title
    \title{tp2pt1}
    
    
    
    
    
    
    
% Pygments definitions
\makeatletter
\def\PY@reset{\let\PY@it=\relax \let\PY@bf=\relax%
    \let\PY@ul=\relax \let\PY@tc=\relax%
    \let\PY@bc=\relax \let\PY@ff=\relax}
\def\PY@tok#1{\csname PY@tok@#1\endcsname}
\def\PY@toks#1+{\ifx\relax#1\empty\else%
    \PY@tok{#1}\expandafter\PY@toks\fi}
\def\PY@do#1{\PY@bc{\PY@tc{\PY@ul{%
    \PY@it{\PY@bf{\PY@ff{#1}}}}}}}
\def\PY#1#2{\PY@reset\PY@toks#1+\relax+\PY@do{#2}}

\@namedef{PY@tok@w}{\def\PY@tc##1{\textcolor[rgb]{0.73,0.73,0.73}{##1}}}
\@namedef{PY@tok@c}{\let\PY@it=\textit\def\PY@tc##1{\textcolor[rgb]{0.24,0.48,0.48}{##1}}}
\@namedef{PY@tok@cp}{\def\PY@tc##1{\textcolor[rgb]{0.61,0.40,0.00}{##1}}}
\@namedef{PY@tok@k}{\let\PY@bf=\textbf\def\PY@tc##1{\textcolor[rgb]{0.00,0.50,0.00}{##1}}}
\@namedef{PY@tok@kp}{\def\PY@tc##1{\textcolor[rgb]{0.00,0.50,0.00}{##1}}}
\@namedef{PY@tok@kt}{\def\PY@tc##1{\textcolor[rgb]{0.69,0.00,0.25}{##1}}}
\@namedef{PY@tok@o}{\def\PY@tc##1{\textcolor[rgb]{0.40,0.40,0.40}{##1}}}
\@namedef{PY@tok@ow}{\let\PY@bf=\textbf\def\PY@tc##1{\textcolor[rgb]{0.67,0.13,1.00}{##1}}}
\@namedef{PY@tok@nb}{\def\PY@tc##1{\textcolor[rgb]{0.00,0.50,0.00}{##1}}}
\@namedef{PY@tok@nf}{\def\PY@tc##1{\textcolor[rgb]{0.00,0.00,1.00}{##1}}}
\@namedef{PY@tok@nc}{\let\PY@bf=\textbf\def\PY@tc##1{\textcolor[rgb]{0.00,0.00,1.00}{##1}}}
\@namedef{PY@tok@nn}{\let\PY@bf=\textbf\def\PY@tc##1{\textcolor[rgb]{0.00,0.00,1.00}{##1}}}
\@namedef{PY@tok@ne}{\let\PY@bf=\textbf\def\PY@tc##1{\textcolor[rgb]{0.80,0.25,0.22}{##1}}}
\@namedef{PY@tok@nv}{\def\PY@tc##1{\textcolor[rgb]{0.10,0.09,0.49}{##1}}}
\@namedef{PY@tok@no}{\def\PY@tc##1{\textcolor[rgb]{0.53,0.00,0.00}{##1}}}
\@namedef{PY@tok@nl}{\def\PY@tc##1{\textcolor[rgb]{0.46,0.46,0.00}{##1}}}
\@namedef{PY@tok@ni}{\let\PY@bf=\textbf\def\PY@tc##1{\textcolor[rgb]{0.44,0.44,0.44}{##1}}}
\@namedef{PY@tok@na}{\def\PY@tc##1{\textcolor[rgb]{0.41,0.47,0.13}{##1}}}
\@namedef{PY@tok@nt}{\let\PY@bf=\textbf\def\PY@tc##1{\textcolor[rgb]{0.00,0.50,0.00}{##1}}}
\@namedef{PY@tok@nd}{\def\PY@tc##1{\textcolor[rgb]{0.67,0.13,1.00}{##1}}}
\@namedef{PY@tok@s}{\def\PY@tc##1{\textcolor[rgb]{0.73,0.13,0.13}{##1}}}
\@namedef{PY@tok@sd}{\let\PY@it=\textit\def\PY@tc##1{\textcolor[rgb]{0.73,0.13,0.13}{##1}}}
\@namedef{PY@tok@si}{\let\PY@bf=\textbf\def\PY@tc##1{\textcolor[rgb]{0.64,0.35,0.47}{##1}}}
\@namedef{PY@tok@se}{\let\PY@bf=\textbf\def\PY@tc##1{\textcolor[rgb]{0.67,0.36,0.12}{##1}}}
\@namedef{PY@tok@sr}{\def\PY@tc##1{\textcolor[rgb]{0.64,0.35,0.47}{##1}}}
\@namedef{PY@tok@ss}{\def\PY@tc##1{\textcolor[rgb]{0.10,0.09,0.49}{##1}}}
\@namedef{PY@tok@sx}{\def\PY@tc##1{\textcolor[rgb]{0.00,0.50,0.00}{##1}}}
\@namedef{PY@tok@m}{\def\PY@tc##1{\textcolor[rgb]{0.40,0.40,0.40}{##1}}}
\@namedef{PY@tok@gh}{\let\PY@bf=\textbf\def\PY@tc##1{\textcolor[rgb]{0.00,0.00,0.50}{##1}}}
\@namedef{PY@tok@gu}{\let\PY@bf=\textbf\def\PY@tc##1{\textcolor[rgb]{0.50,0.00,0.50}{##1}}}
\@namedef{PY@tok@gd}{\def\PY@tc##1{\textcolor[rgb]{0.63,0.00,0.00}{##1}}}
\@namedef{PY@tok@gi}{\def\PY@tc##1{\textcolor[rgb]{0.00,0.52,0.00}{##1}}}
\@namedef{PY@tok@gr}{\def\PY@tc##1{\textcolor[rgb]{0.89,0.00,0.00}{##1}}}
\@namedef{PY@tok@ge}{\let\PY@it=\textit}
\@namedef{PY@tok@gs}{\let\PY@bf=\textbf}
\@namedef{PY@tok@gp}{\let\PY@bf=\textbf\def\PY@tc##1{\textcolor[rgb]{0.00,0.00,0.50}{##1}}}
\@namedef{PY@tok@go}{\def\PY@tc##1{\textcolor[rgb]{0.44,0.44,0.44}{##1}}}
\@namedef{PY@tok@gt}{\def\PY@tc##1{\textcolor[rgb]{0.00,0.27,0.87}{##1}}}
\@namedef{PY@tok@err}{\def\PY@bc##1{{\setlength{\fboxsep}{\string -\fboxrule}\fcolorbox[rgb]{1.00,0.00,0.00}{1,1,1}{\strut ##1}}}}
\@namedef{PY@tok@kc}{\let\PY@bf=\textbf\def\PY@tc##1{\textcolor[rgb]{0.00,0.50,0.00}{##1}}}
\@namedef{PY@tok@kd}{\let\PY@bf=\textbf\def\PY@tc##1{\textcolor[rgb]{0.00,0.50,0.00}{##1}}}
\@namedef{PY@tok@kn}{\let\PY@bf=\textbf\def\PY@tc##1{\textcolor[rgb]{0.00,0.50,0.00}{##1}}}
\@namedef{PY@tok@kr}{\let\PY@bf=\textbf\def\PY@tc##1{\textcolor[rgb]{0.00,0.50,0.00}{##1}}}
\@namedef{PY@tok@bp}{\def\PY@tc##1{\textcolor[rgb]{0.00,0.50,0.00}{##1}}}
\@namedef{PY@tok@fm}{\def\PY@tc##1{\textcolor[rgb]{0.00,0.00,1.00}{##1}}}
\@namedef{PY@tok@vc}{\def\PY@tc##1{\textcolor[rgb]{0.10,0.09,0.49}{##1}}}
\@namedef{PY@tok@vg}{\def\PY@tc##1{\textcolor[rgb]{0.10,0.09,0.49}{##1}}}
\@namedef{PY@tok@vi}{\def\PY@tc##1{\textcolor[rgb]{0.10,0.09,0.49}{##1}}}
\@namedef{PY@tok@vm}{\def\PY@tc##1{\textcolor[rgb]{0.10,0.09,0.49}{##1}}}
\@namedef{PY@tok@sa}{\def\PY@tc##1{\textcolor[rgb]{0.73,0.13,0.13}{##1}}}
\@namedef{PY@tok@sb}{\def\PY@tc##1{\textcolor[rgb]{0.73,0.13,0.13}{##1}}}
\@namedef{PY@tok@sc}{\def\PY@tc##1{\textcolor[rgb]{0.73,0.13,0.13}{##1}}}
\@namedef{PY@tok@dl}{\def\PY@tc##1{\textcolor[rgb]{0.73,0.13,0.13}{##1}}}
\@namedef{PY@tok@s2}{\def\PY@tc##1{\textcolor[rgb]{0.73,0.13,0.13}{##1}}}
\@namedef{PY@tok@sh}{\def\PY@tc##1{\textcolor[rgb]{0.73,0.13,0.13}{##1}}}
\@namedef{PY@tok@s1}{\def\PY@tc##1{\textcolor[rgb]{0.73,0.13,0.13}{##1}}}
\@namedef{PY@tok@mb}{\def\PY@tc##1{\textcolor[rgb]{0.40,0.40,0.40}{##1}}}
\@namedef{PY@tok@mf}{\def\PY@tc##1{\textcolor[rgb]{0.40,0.40,0.40}{##1}}}
\@namedef{PY@tok@mh}{\def\PY@tc##1{\textcolor[rgb]{0.40,0.40,0.40}{##1}}}
\@namedef{PY@tok@mi}{\def\PY@tc##1{\textcolor[rgb]{0.40,0.40,0.40}{##1}}}
\@namedef{PY@tok@il}{\def\PY@tc##1{\textcolor[rgb]{0.40,0.40,0.40}{##1}}}
\@namedef{PY@tok@mo}{\def\PY@tc##1{\textcolor[rgb]{0.40,0.40,0.40}{##1}}}
\@namedef{PY@tok@ch}{\let\PY@it=\textit\def\PY@tc##1{\textcolor[rgb]{0.24,0.48,0.48}{##1}}}
\@namedef{PY@tok@cm}{\let\PY@it=\textit\def\PY@tc##1{\textcolor[rgb]{0.24,0.48,0.48}{##1}}}
\@namedef{PY@tok@cpf}{\let\PY@it=\textit\def\PY@tc##1{\textcolor[rgb]{0.24,0.48,0.48}{##1}}}
\@namedef{PY@tok@c1}{\let\PY@it=\textit\def\PY@tc##1{\textcolor[rgb]{0.24,0.48,0.48}{##1}}}
\@namedef{PY@tok@cs}{\let\PY@it=\textit\def\PY@tc##1{\textcolor[rgb]{0.24,0.48,0.48}{##1}}}

\def\PYZbs{\char`\\}
\def\PYZus{\char`\_}
\def\PYZob{\char`\{}
\def\PYZcb{\char`\}}
\def\PYZca{\char`\^}
\def\PYZam{\char`\&}
\def\PYZlt{\char`\<}
\def\PYZgt{\char`\>}
\def\PYZsh{\char`\#}
\def\PYZpc{\char`\%}
\def\PYZdl{\char`\$}
\def\PYZhy{\char`\-}
\def\PYZsq{\char`\'}
\def\PYZdq{\char`\"}
\def\PYZti{\char`\~}
% for compatibility with earlier versions
\def\PYZat{@}
\def\PYZlb{[}
\def\PYZrb{]}
\makeatother


    % For linebreaks inside Verbatim environment from package fancyvrb.
    \makeatletter
        \newbox\Wrappedcontinuationbox
        \newbox\Wrappedvisiblespacebox
        \newcommand*\Wrappedvisiblespace {\textcolor{red}{\textvisiblespace}}
        \newcommand*\Wrappedcontinuationsymbol {\textcolor{red}{\llap{\tiny$\m@th\hookrightarrow$}}}
        \newcommand*\Wrappedcontinuationindent {3ex }
        \newcommand*\Wrappedafterbreak {\kern\Wrappedcontinuationindent\copy\Wrappedcontinuationbox}
        % Take advantage of the already applied Pygments mark-up to insert
        % potential linebreaks for TeX processing.
        %        {, <, #, %, $, ' and ": go to next line.
        %        _, }, ^, &, >, - and ~: stay at end of broken line.
        % Use of \textquotesingle for straight quote.
        \newcommand*\Wrappedbreaksatspecials {%
            \def\PYGZus{\discretionary{\char`\_}{\Wrappedafterbreak}{\char`\_}}%
            \def\PYGZob{\discretionary{}{\Wrappedafterbreak\char`\{}{\char`\{}}%
            \def\PYGZcb{\discretionary{\char`\}}{\Wrappedafterbreak}{\char`\}}}%
            \def\PYGZca{\discretionary{\char`\^}{\Wrappedafterbreak}{\char`\^}}%
            \def\PYGZam{\discretionary{\char`\&}{\Wrappedafterbreak}{\char`\&}}%
            \def\PYGZlt{\discretionary{}{\Wrappedafterbreak\char`\<}{\char`\<}}%
            \def\PYGZgt{\discretionary{\char`\>}{\Wrappedafterbreak}{\char`\>}}%
            \def\PYGZsh{\discretionary{}{\Wrappedafterbreak\char`\#}{\char`\#}}%
            \def\PYGZpc{\discretionary{}{\Wrappedafterbreak\char`\%}{\char`\%}}%
            \def\PYGZdl{\discretionary{}{\Wrappedafterbreak\char`\$}{\char`\$}}%
            \def\PYGZhy{\discretionary{\char`\-}{\Wrappedafterbreak}{\char`\-}}%
            \def\PYGZsq{\discretionary{}{\Wrappedafterbreak\textquotesingle}{\textquotesingle}}%
            \def\PYGZdq{\discretionary{}{\Wrappedafterbreak\char`\"}{\char`\"}}%
            \def\PYGZti{\discretionary{\char`\~}{\Wrappedafterbreak}{\char`\~}}%
        }
        % Some characters . , ; ? ! / are not pygmentized.
        % This macro makes them "active" and they will insert potential linebreaks
        \newcommand*\Wrappedbreaksatpunct {%
            \lccode`\~`\.\lowercase{\def~}{\discretionary{\hbox{\char`\.}}{\Wrappedafterbreak}{\hbox{\char`\.}}}%
            \lccode`\~`\,\lowercase{\def~}{\discretionary{\hbox{\char`\,}}{\Wrappedafterbreak}{\hbox{\char`\,}}}%
            \lccode`\~`\;\lowercase{\def~}{\discretionary{\hbox{\char`\;}}{\Wrappedafterbreak}{\hbox{\char`\;}}}%
            \lccode`\~`\:\lowercase{\def~}{\discretionary{\hbox{\char`\:}}{\Wrappedafterbreak}{\hbox{\char`\:}}}%
            \lccode`\~`\?\lowercase{\def~}{\discretionary{\hbox{\char`\?}}{\Wrappedafterbreak}{\hbox{\char`\?}}}%
            \lccode`\~`\!\lowercase{\def~}{\discretionary{\hbox{\char`\!}}{\Wrappedafterbreak}{\hbox{\char`\!}}}%
            \lccode`\~`\/\lowercase{\def~}{\discretionary{\hbox{\char`\/}}{\Wrappedafterbreak}{\hbox{\char`\/}}}%
            \catcode`\.\active
            \catcode`\,\active
            \catcode`\;\active
            \catcode`\:\active
            \catcode`\?\active
            \catcode`\!\active
            \catcode`\/\active
            \lccode`\~`\~
        }
    \makeatother

    \let\OriginalVerbatim=\Verbatim
    \makeatletter
    \renewcommand{\Verbatim}[1][1]{%
        %\parskip\z@skip
        \sbox\Wrappedcontinuationbox {\Wrappedcontinuationsymbol}%
        \sbox\Wrappedvisiblespacebox {\FV@SetupFont\Wrappedvisiblespace}%
        \def\FancyVerbFormatLine ##1{\hsize\linewidth
            \vtop{\raggedright\hyphenpenalty\z@\exhyphenpenalty\z@
                \doublehyphendemerits\z@\finalhyphendemerits\z@
                \strut ##1\strut}%
        }%
        % If the linebreak is at a space, the latter will be displayed as visible
        % space at end of first line, and a continuation symbol starts next line.
        % Stretch/shrink are however usually zero for typewriter font.
        \def\FV@Space {%
            \nobreak\hskip\z@ plus\fontdimen3\font minus\fontdimen4\font
            \discretionary{\copy\Wrappedvisiblespacebox}{\Wrappedafterbreak}
            {\kern\fontdimen2\font}%
        }%

        % Allow breaks at special characters using \PYG... macros.
        \Wrappedbreaksatspecials
        % Breaks at punctuation characters . , ; ? ! and / need catcode=\active
        \OriginalVerbatim[#1,codes*=\Wrappedbreaksatpunct]%
    }
    \makeatother

    % Exact colors from NB
    \definecolor{incolor}{HTML}{303F9F}
    \definecolor{outcolor}{HTML}{D84315}
    \definecolor{cellborder}{HTML}{CFCFCF}
    \definecolor{cellbackground}{HTML}{F7F7F7}

    % prompt
    \makeatletter
    \newcommand{\boxspacing}{\kern\kvtcb@left@rule\kern\kvtcb@boxsep}
    \makeatother
    \newcommand{\prompt}[4]{
        {\ttfamily\llap{{\color{#2}[#3]:\hspace{3pt}#4}}\vspace{-\baselineskip}}
    }
    

    
    % Prevent overflowing lines due to hard-to-break entities
    \sloppy
    % Setup hyperref package
    \hypersetup{
      breaklinks=true,  % so long urls are correctly broken across lines
      colorlinks=true,
      urlcolor=urlcolor,
      linkcolor=linkcolor,
      citecolor=citecolor,
      }
    % Slightly bigger margins than the latex defaults
    
    \geometry{verbose,tmargin=1in,bmargin=1in,lmargin=1in,rmargin=1in}
    
    

\begin{document}
    
    \maketitle
    
    

    
    \hypertarget{outubro-8-2024}{%
\subsubsection{Outubro, 8, 2024}\label{outubro-8-2024}}

\hypertarget{tp2---grupo-20}{%
\subsubsection{TP2 - Grupo 20}\label{tp2---grupo-20}}

Afonso Martins Campos Fernandes - A102940

Luís Felipe Pinheiro Silva - A105530

\hypertarget{exercicio-1}{%
\subparagraph{Exercicio 1:}\label{exercicio-1}}

\#\#\#\#\# Considere a descrição da cifra A5/1 que consta no documento
+Lógica Computacional: a Cifra A5/1 Informação complementar pode ser
obtida no artigo da Wikipedia.

\hypertarget{a-definir-e-codificar-em-z3-e-usando-o-tipo-bitvec-para-modelar-a-informauxe7uxe3o-uma-fsm-que-descreva-o-gerador.}{%
\subparagraph{a) Definir e codificar, em z3 e usando o tipo BitVec para
modelar a informação, uma FSM que descreva o
gerador.}\label{a-definir-e-codificar-em-z3-e-usando-o-tipo-bitvec-para-modelar-a-informauxe7uxe3o-uma-fsm-que-descreva-o-gerador.}}

    \begin{tcolorbox}[breakable, size=fbox, boxrule=1pt, pad at break*=1mm,colback=cellbackground, colframe=cellborder]
\prompt{In}{incolor}{190}{\boxspacing}
\begin{Verbatim}[commandchars=\\\{\}]
\PY{k+kn}{from} \PY{n+nn}{z3} \PY{k+kn}{import} \PY{o}{*}
\PY{k+kn}{import} \PY{n+nn}{random}
\PY{k+kn}{from} \PY{n+nn}{random} \PY{k+kn}{import} \PY{n}{getrandbits}

\PY{c+c1}{\PYZsh{} Definir os LFSRs como BitVecs com os tamanhos adequados}
\PY{n}{LFSR1} \PY{o}{=} \PY{n}{BitVec}\PY{p}{(}\PY{l+s+s1}{\PYZsq{}}\PY{l+s+s1}{LFSR1}\PY{l+s+s1}{\PYZsq{}}\PY{p}{,} \PY{l+m+mi}{19}\PY{p}{)}
\PY{n}{LFSR2} \PY{o}{=} \PY{n}{BitVec}\PY{p}{(}\PY{l+s+s1}{\PYZsq{}}\PY{l+s+s1}{LFSR2}\PY{l+s+s1}{\PYZsq{}}\PY{p}{,} \PY{l+m+mi}{22}\PY{p}{)}
\PY{n}{LFSR3} \PY{o}{=} \PY{n}{BitVec}\PY{p}{(}\PY{l+s+s1}{\PYZsq{}}\PY{l+s+s1}{LFSR3}\PY{l+s+s1}{\PYZsq{}}\PY{p}{,} \PY{l+m+mi}{23}\PY{p}{)}

\PY{c+c1}{\PYZsh{} Funções de atualização para cada LFSR}
\PY{k}{def} \PY{n+nf}{lfsr1\PYZus{}seguinte}\PY{p}{(}\PY{n}{LFSR1}\PY{p}{)}\PY{p}{:}
    \PY{n}{f} \PY{o}{=} \PY{n}{Extract}\PY{p}{(}\PY{l+m+mi}{18}\PY{p}{,} \PY{l+m+mi}{18}\PY{p}{,} \PY{n}{LFSR1}\PY{p}{)} \PY{o}{\PYZca{}} \PY{n}{Extract}\PY{p}{(}\PY{l+m+mi}{17}\PY{p}{,} \PY{l+m+mi}{17}\PY{p}{,} \PY{n}{LFSR1}\PY{p}{)} \PY{o}{\PYZca{}} \PY{n}{Extract}\PY{p}{(}\PY{l+m+mi}{16}\PY{p}{,} \PY{l+m+mi}{16}\PY{p}{,} \PY{n}{LFSR1}\PY{p}{)} \PY{o}{\PYZca{}} \PY{n}{Extract}\PY{p}{(}\PY{l+m+mi}{13}\PY{p}{,} \PY{l+m+mi}{13}\PY{p}{,} \PY{n}{LFSR1}\PY{p}{)}
    \PY{k}{return} \PY{n}{Concat}\PY{p}{(}\PY{n}{f}\PY{p}{,} \PY{n}{Extract}\PY{p}{(}\PY{l+m+mi}{18}\PY{p}{,} \PY{l+m+mi}{1}\PY{p}{,} \PY{n}{LFSR1}\PY{p}{)}\PY{p}{)}

\PY{k}{def} \PY{n+nf}{lfsr2\PYZus{}seguinte}\PY{p}{(}\PY{n}{LFSR2}\PY{p}{)}\PY{p}{:}
    \PY{n}{f} \PY{o}{=} \PY{n}{Extract}\PY{p}{(}\PY{l+m+mi}{21}\PY{p}{,} \PY{l+m+mi}{21}\PY{p}{,} \PY{n}{LFSR2}\PY{p}{)} \PY{o}{\PYZca{}} \PY{n}{Extract}\PY{p}{(}\PY{l+m+mi}{20}\PY{p}{,} \PY{l+m+mi}{20}\PY{p}{,} \PY{n}{LFSR2}\PY{p}{)}
    \PY{k}{return} \PY{n}{Concat}\PY{p}{(}\PY{n}{f}\PY{p}{,} \PY{n}{Extract}\PY{p}{(}\PY{l+m+mi}{21}\PY{p}{,} \PY{l+m+mi}{1}\PY{p}{,} \PY{n}{LFSR2}\PY{p}{)}\PY{p}{)}

\PY{k}{def} \PY{n+nf}{lfsr3\PYZus{}seguinte}\PY{p}{(}\PY{n}{LFSR3}\PY{p}{)}\PY{p}{:}
    \PY{n}{f} \PY{o}{=} \PY{n}{Extract}\PY{p}{(}\PY{l+m+mi}{22}\PY{p}{,} \PY{l+m+mi}{22}\PY{p}{,} \PY{n}{LFSR3}\PY{p}{)} \PY{o}{\PYZca{}} \PY{n}{Extract}\PY{p}{(}\PY{l+m+mi}{21}\PY{p}{,} \PY{l+m+mi}{21}\PY{p}{,} \PY{n}{LFSR3}\PY{p}{)} \PY{o}{\PYZca{}} \PY{n}{Extract}\PY{p}{(}\PY{l+m+mi}{20}\PY{p}{,} \PY{l+m+mi}{20}\PY{p}{,} \PY{n}{LFSR3}\PY{p}{)} \PY{o}{\PYZca{}} \PY{n}{Extract}\PY{p}{(}\PY{l+m+mi}{7}\PY{p}{,} \PY{l+m+mi}{7}\PY{p}{,} \PY{n}{LFSR3}\PY{p}{)}
    \PY{k}{return} \PY{n}{Concat}\PY{p}{(}\PY{n}{f}\PY{p}{,} \PY{n}{Extract}\PY{p}{(}\PY{l+m+mi}{22}\PY{p}{,} \PY{l+m+mi}{1}\PY{p}{,} \PY{n}{LFSR3}\PY{p}{)}\PY{p}{)}

\PY{c+c1}{\PYZsh{} Bits de clock para cada LFSR}
\PY{n}{cBit1} \PY{o}{=} \PY{n}{Extract}\PY{p}{(}\PY{l+m+mi}{8}\PY{p}{,} \PY{l+m+mi}{8}\PY{p}{,} \PY{n}{LFSR1}\PY{p}{)}
\PY{n}{cBit2} \PY{o}{=} \PY{n}{Extract}\PY{p}{(}\PY{l+m+mi}{10}\PY{p}{,} \PY{l+m+mi}{10}\PY{p}{,} \PY{n}{LFSR2}\PY{p}{)}
\PY{n}{cBit3} \PY{o}{=} \PY{n}{Extract}\PY{p}{(}\PY{l+m+mi}{10}\PY{p}{,} \PY{l+m+mi}{10}\PY{p}{,} \PY{n}{LFSR3}\PY{p}{)}

\PY{c+c1}{\PYZsh{} Função de clock majoritário}
\PY{k}{def} \PY{n+nf}{majority}\PY{p}{(}\PY{n}{b1}\PY{p}{,} \PY{n}{b2}\PY{p}{,} \PY{n}{b3}\PY{p}{)}\PY{p}{:}
    \PY{k}{return} \PY{n}{If}\PY{p}{(}\PY{n}{b1} \PY{o}{+} \PY{n}{b2} \PY{o}{+} \PY{n}{b3} \PY{o}{\PYZgt{}} \PY{l+m+mi}{1}\PY{p}{,} \PY{n}{BitVecVal}\PY{p}{(}\PY{l+m+mi}{1}\PY{p}{,} \PY{l+m+mi}{1}\PY{p}{)}\PY{p}{,} \PY{n}{BitVecVal}\PY{p}{(}\PY{l+m+mi}{0}\PY{p}{,} \PY{l+m+mi}{1}\PY{p}{)}\PY{p}{)}

\PY{c+c1}{\PYZsh{} Calcular o bit majoritário}
\PY{n}{majority\PYZus{}bit} \PY{o}{=} \PY{n}{majority}\PY{p}{(}\PY{n}{cBit1}\PY{p}{,} \PY{n}{cBit2}\PY{p}{,} \PY{n}{cBit3}\PY{p}{)}

\PY{c+c1}{\PYZsh{} Atualizar os LFSRs com base no bit de clock majoritário}
\PY{n}{next\PYZus{}LFSR1} \PY{o}{=} \PY{n}{If}\PY{p}{(}\PY{n}{cBit1} \PY{o}{==} \PY{n}{majority\PYZus{}bit}\PY{p}{,} \PY{n}{lfsr1\PYZus{}seguinte}\PY{p}{(}\PY{n}{LFSR1}\PY{p}{)}\PY{p}{,} \PY{n}{LFSR1}\PY{p}{)}
\PY{n}{next\PYZus{}LFSR2} \PY{o}{=} \PY{n}{If}\PY{p}{(}\PY{n}{cBit2} \PY{o}{==} \PY{n}{majority\PYZus{}bit}\PY{p}{,} \PY{n}{lfsr2\PYZus{}seguinte}\PY{p}{(}\PY{n}{LFSR2}\PY{p}{)}\PY{p}{,} \PY{n}{LFSR2}\PY{p}{)}
\PY{n}{next\PYZus{}LFSR3} \PY{o}{=} \PY{n}{If}\PY{p}{(}\PY{n}{cBit3} \PY{o}{==} \PY{n}{majority\PYZus{}bit}\PY{p}{,} \PY{n}{lfsr3\PYZus{}seguinte}\PY{p}{(}\PY{n}{LFSR3}\PY{p}{)}\PY{p}{,} \PY{n}{LFSR3}\PY{p}{)}

\PY{c+c1}{\PYZsh{} Solver para testar a transição}
\PY{n}{solver} \PY{o}{=} \PY{n}{Solver}\PY{p}{(}\PY{p}{)}
\PY{n}{solver}\PY{o}{.}\PY{n}{add}\PY{p}{(}\PY{n}{LFSR1} \PY{o}{==} \PY{n}{BitVecVal}\PY{p}{(}\PY{n}{getrandbits}\PY{p}{(}\PY{l+m+mi}{19}\PY{p}{)}\PY{p}{,} \PY{l+m+mi}{19}\PY{p}{)}\PY{p}{)}
\PY{n}{solver}\PY{o}{.}\PY{n}{add}\PY{p}{(}\PY{n}{LFSR2} \PY{o}{==} \PY{n}{BitVecVal}\PY{p}{(}\PY{n}{getrandbits}\PY{p}{(}\PY{l+m+mi}{22}\PY{p}{)}\PY{p}{,} \PY{l+m+mi}{22}\PY{p}{)}\PY{p}{)}
\PY{n}{solver}\PY{o}{.}\PY{n}{add}\PY{p}{(}\PY{n}{LFSR3} \PY{o}{==} \PY{n}{BitVecVal}\PY{p}{(}\PY{n}{getrandbits}\PY{p}{(}\PY{l+m+mi}{23}\PY{p}{)}\PY{p}{,} \PY{l+m+mi}{23}\PY{p}{)}\PY{p}{)}

\PY{c+c1}{\PYZsh{} Verificação de estados possíveis}
\PY{k}{if} \PY{n}{solver}\PY{o}{.}\PY{n}{check}\PY{p}{(}\PY{p}{)} \PY{o}{==} \PY{n}{sat}\PY{p}{:}
    \PY{n}{modelo} \PY{o}{=} \PY{n}{solver}\PY{o}{.}\PY{n}{model}\PY{p}{(}\PY{p}{)}
    \PY{n+nb}{print}\PY{p}{(}\PY{l+s+s2}{\PYZdq{}}\PY{l+s+s2}{Estado inicial:}\PY{l+s+s2}{\PYZdq{}}\PY{p}{)}
    \PY{n+nb}{print}\PY{p}{(}\PY{l+s+s2}{\PYZdq{}}\PY{l+s+s2}{LFSR1:}\PY{l+s+s2}{\PYZdq{}}\PY{p}{,} \PY{n}{modelo}\PY{p}{[}\PY{n}{LFSR1}\PY{p}{]}\PY{p}{)}
    \PY{n+nb}{print}\PY{p}{(}\PY{l+s+s2}{\PYZdq{}}\PY{l+s+s2}{LFSR2:}\PY{l+s+s2}{\PYZdq{}}\PY{p}{,} \PY{n}{modelo}\PY{p}{[}\PY{n}{LFSR2}\PY{p}{]}\PY{p}{)}
    \PY{n+nb}{print}\PY{p}{(}\PY{l+s+s2}{\PYZdq{}}\PY{l+s+s2}{LFSR3:}\PY{l+s+s2}{\PYZdq{}}\PY{p}{,} \PY{n}{modelo}\PY{p}{[}\PY{n}{LFSR3}\PY{p}{]}\PY{p}{)}
    
    \PY{c+c1}{\PYZsh{} Simulação de uma transição}
    \PY{n+nb}{print}\PY{p}{(}\PY{l+s+s2}{\PYZdq{}}\PY{l+s+se}{\PYZbs{}n}\PY{l+s+s2}{Próximo Estado:}\PY{l+s+s2}{\PYZdq{}}\PY{p}{)}
    \PY{n+nb}{print}\PY{p}{(}\PY{l+s+s2}{\PYZdq{}}\PY{l+s+s2}{Next\PYZus{}LFSR1:}\PY{l+s+s2}{\PYZdq{}}\PY{p}{,} \PY{n}{modelo}\PY{o}{.}\PY{n}{evaluate}\PY{p}{(}\PY{n}{next\PYZus{}LFSR1}\PY{p}{)}\PY{p}{)}
    \PY{n+nb}{print}\PY{p}{(}\PY{l+s+s2}{\PYZdq{}}\PY{l+s+s2}{Next\PYZus{}LFSR2:}\PY{l+s+s2}{\PYZdq{}}\PY{p}{,} \PY{n}{modelo}\PY{o}{.}\PY{n}{evaluate}\PY{p}{(}\PY{n}{next\PYZus{}LFSR2}\PY{p}{)}\PY{p}{)}
    \PY{n+nb}{print}\PY{p}{(}\PY{l+s+s2}{\PYZdq{}}\PY{l+s+s2}{Next\PYZus{}LFSR3:}\PY{l+s+s2}{\PYZdq{}}\PY{p}{,} \PY{n}{modelo}\PY{o}{.}\PY{n}{evaluate}\PY{p}{(}\PY{n}{next\PYZus{}LFSR3}\PY{p}{)}\PY{p}{)}
\PY{k}{else}\PY{p}{:}
    \PY{n+nb}{print}\PY{p}{(}\PY{l+s+s2}{\PYZdq{}}\PY{l+s+s2}{Nenhuma solução foi encontrada.}\PY{l+s+s2}{\PYZdq{}}\PY{p}{)}
\end{Verbatim}
\end{tcolorbox}

    \begin{Verbatim}[commandchars=\\\{\}]
Estado inicial:
LFSR1: 429273
LFSR2: 290254
LFSR3: 6649777

Próximo Estado:
Next\_LFSR1: 429273
Next\_LFSR2: 145127
Next\_LFSR3: 7519192
    \end{Verbatim}

    \hypertarget{b-considere-as-seguintes-propriedades-de-erro}{%
\subsubsection{b) Considere as seguintes propriedades de
erro:}\label{b-considere-as-seguintes-propriedades-de-erro}}

\hypertarget{i-ocorruxeancia-de-um-burst-0tt-zeros-que-ocorre-em-2t-passos-ou-menos.}{%
\subsubsection{i) Ocorrência de um ``burst'' 0\^{}t(t-zeros) que ocorre
em 2\^{}t passos ou
menos.}\label{i-ocorruxeancia-de-um-burst-0tt-zeros-que-ocorre-em-2t-passos-ou-menos.}}

Tente codificar estas propriedade e cerificar se são acessíveis a partir
de um estado inicial aleatoriamente gerado

    \begin{tcolorbox}[breakable, size=fbox, boxrule=1pt, pad at break*=1mm,colback=cellbackground, colframe=cellborder]
\prompt{In}{incolor}{191}{\boxspacing}
\begin{Verbatim}[commandchars=\\\{\}]
\PY{k+kn}{from} \PY{n+nn}{z3} \PY{k+kn}{import} \PY{o}{*}
\PY{k+kn}{from} \PY{n+nn}{random} \PY{k+kn}{import} \PY{n}{getrandbits}

\PY{c+c1}{\PYZsh{} Parâmetros da propriedade de erro}
\PY{n}{t} \PY{o}{=} \PY{l+m+mi}{3}  \PY{c+c1}{\PYZsh{} Número de zeros consecutivos (tamanho do \PYZdq{}burst\PYZdq{})}
\PY{n}{passos} \PY{o}{=} \PY{l+m+mi}{2} \PY{o}{*}\PY{o}{*} \PY{n}{t}  \PY{c+c1}{\PYZsh{} Número máximo de passos permitidos para encontrar o \PYZdq{}burst\PYZdq{}}

\PY{c+c1}{\PYZsh{} Configurar o solver}
\PY{n}{solver} \PY{o}{=} \PY{n}{Solver}\PY{p}{(}\PY{p}{)}

\PY{c+c1}{\PYZsh{} Inicializar os LFSRs com estados aleatórios}
\PY{n}{curLFSR1} \PY{o}{=} \PY{n}{BitVecVal}\PY{p}{(}\PY{n}{getrandbits}\PY{p}{(}\PY{l+m+mi}{19}\PY{p}{)}\PY{p}{,} \PY{l+m+mi}{19}\PY{p}{)}
\PY{n}{curLFSR2} \PY{o}{=} \PY{n}{BitVecVal}\PY{p}{(}\PY{n}{getrandbits}\PY{p}{(}\PY{l+m+mi}{22}\PY{p}{)}\PY{p}{,} \PY{l+m+mi}{22}\PY{p}{)}
\PY{n}{curLFSR3} \PY{o}{=} \PY{n}{BitVecVal}\PY{p}{(}\PY{n}{getrandbits}\PY{p}{(}\PY{l+m+mi}{23}\PY{p}{)}\PY{p}{,} \PY{l+m+mi}{23}\PY{p}{)}

\PY{n}{countZeros} \PY{o}{=} \PY{n}{BitVecVal}\PY{p}{(}\PY{l+m+mi}{0}\PY{p}{,} \PY{l+m+mi}{32}\PY{p}{)}  \PY{c+c1}{\PYZsh{} Inicializar o contador de zeros consecutivos}
\PY{n}{outputs} \PY{o}{=} \PY{p}{[}\PY{p}{]}

\PY{c+c1}{\PYZsh{} Loop para simular os passos e procurar por um \PYZdq{}burst\PYZdq{} de zeros}
\PY{k}{for} \PY{n}{passo} \PY{o+ow}{in} \PY{n+nb}{range}\PY{p}{(}\PY{n}{passos}\PY{p}{)}\PY{p}{:}
    \PY{c+c1}{\PYZsh{} Calcular o bit majoritário}
    \PY{n}{majority\PYZus{}bit} \PY{o}{=} \PY{n}{majority}\PY{p}{(}\PY{n}{Extract}\PY{p}{(}\PY{l+m+mi}{8}\PY{p}{,} \PY{l+m+mi}{8}\PY{p}{,} \PY{n}{curLFSR1}\PY{p}{)}\PY{p}{,} \PY{n}{Extract}\PY{p}{(}\PY{l+m+mi}{10}\PY{p}{,} \PY{l+m+mi}{10}\PY{p}{,} \PY{n}{curLFSR2}\PY{p}{)}\PY{p}{,} \PY{n}{Extract}\PY{p}{(}\PY{l+m+mi}{10}\PY{p}{,} \PY{l+m+mi}{10}\PY{p}{,} \PY{n}{curLFSR3}\PY{p}{)}\PY{p}{)}
    
    \PY{c+c1}{\PYZsh{} Atualizar os LFSRs com base no bit de clock majoritário}
    \PY{n}{next\PYZus{}LFSR1} \PY{o}{=} \PY{n}{If}\PY{p}{(}\PY{n}{Extract}\PY{p}{(}\PY{l+m+mi}{8}\PY{p}{,} \PY{l+m+mi}{8}\PY{p}{,} \PY{n}{curLFSR1}\PY{p}{)} \PY{o}{==} \PY{n}{majority\PYZus{}bit}\PY{p}{,} \PY{n}{lfsr1\PYZus{}seguinte}\PY{p}{(}\PY{n}{curLFSR1}\PY{p}{)}\PY{p}{,} \PY{n}{curLFSR1}\PY{p}{)}
    \PY{n}{next\PYZus{}LFSR2} \PY{o}{=} \PY{n}{If}\PY{p}{(}\PY{n}{Extract}\PY{p}{(}\PY{l+m+mi}{10}\PY{p}{,} \PY{l+m+mi}{10}\PY{p}{,} \PY{n}{curLFSR2}\PY{p}{)} \PY{o}{==} \PY{n}{majority\PYZus{}bit}\PY{p}{,} \PY{n}{lfsr2\PYZus{}seguinte}\PY{p}{(}\PY{n}{curLFSR2}\PY{p}{)}\PY{p}{,} \PY{n}{curLFSR2}\PY{p}{)}
    \PY{n}{next\PYZus{}LFSR3} \PY{o}{=} \PY{n}{If}\PY{p}{(}\PY{n}{Extract}\PY{p}{(}\PY{l+m+mi}{10}\PY{p}{,} \PY{l+m+mi}{10}\PY{p}{,} \PY{n}{curLFSR3}\PY{p}{)} \PY{o}{==} \PY{n}{majority\PYZus{}bit}\PY{p}{,} \PY{n}{lfsr3\PYZus{}seguinte}\PY{p}{(}\PY{n}{curLFSR3}\PY{p}{)}\PY{p}{,} \PY{n}{curLFSR3}\PY{p}{)}
    
    \PY{c+c1}{\PYZsh{} Atualizar o estado atual dos LFSRs}
    \PY{n}{curLFSR1}\PY{p}{,} \PY{n}{curLFSR2}\PY{p}{,} \PY{n}{curLFSR3} \PY{o}{=} \PY{n}{next\PYZus{}LFSR1}\PY{p}{,} \PY{n}{next\PYZus{}LFSR2}\PY{p}{,} \PY{n}{next\PYZus{}LFSR3}
    
    \PY{c+c1}{\PYZsh{} Extrair o bit menos significativo de cada LFSR e calcular o bit de saída}
    \PY{n}{output\PYZus{}bit} \PY{o}{=} \PY{n}{Extract}\PY{p}{(}\PY{l+m+mi}{0}\PY{p}{,} \PY{l+m+mi}{0}\PY{p}{,} \PY{n}{curLFSR1}\PY{p}{)} \PY{o}{\PYZca{}} \PY{n}{Extract}\PY{p}{(}\PY{l+m+mi}{0}\PY{p}{,} \PY{l+m+mi}{0}\PY{p}{,} \PY{n}{curLFSR2}\PY{p}{)} \PY{o}{\PYZca{}} \PY{n}{Extract}\PY{p}{(}\PY{l+m+mi}{0}\PY{p}{,} \PY{l+m+mi}{0}\PY{p}{,} \PY{n}{curLFSR3}\PY{p}{)}
    \PY{n}{outputs}\PY{o}{.}\PY{n}{append}\PY{p}{(}\PY{n}{output\PYZus{}bit}\PY{p}{)}
    
    \PY{c+c1}{\PYZsh{} Verificar se o bit de saída é zero e atualizar o contador de zeros}
    \PY{n}{isZero} \PY{o}{=} \PY{n}{output\PYZus{}bit} \PY{o}{==} \PY{n}{BitVecVal}\PY{p}{(}\PY{l+m+mi}{0}\PY{p}{,} \PY{l+m+mi}{1}\PY{p}{)}
    \PY{n}{countZeros} \PY{o}{=} \PY{n}{If}\PY{p}{(}\PY{n}{isZero}\PY{p}{,} \PY{n}{countZeros} \PY{o}{+} \PY{l+m+mi}{1}\PY{p}{,} \PY{n}{BitVecVal}\PY{p}{(}\PY{l+m+mi}{0}\PY{p}{,} \PY{l+m+mi}{32}\PY{p}{)}\PY{p}{)}  \PY{c+c1}{\PYZsh{} Reinicia o contador se não for zero}

    \PY{c+c1}{\PYZsh{} Condição de \PYZdq{}burst\PYZdq{} de zeros: `t` zeros consecutivos}
    \PY{n}{solver}\PY{o}{.}\PY{n}{add}\PY{p}{(}\PY{n}{countZeros} \PY{o}{\PYZlt{}}\PY{o}{=} \PY{n}{t}\PY{p}{)}

\PY{c+c1}{\PYZsh{} Restrição para garantir pelo menos um bit zero e um bit um no output}
\PY{n}{solver}\PY{o}{.}\PY{n}{add}\PY{p}{(}\PY{n}{Or}\PY{p}{(}\PY{p}{[}\PY{n}{output} \PY{o}{==} \PY{n}{BitVecVal}\PY{p}{(}\PY{l+m+mi}{0}\PY{p}{,} \PY{l+m+mi}{1}\PY{p}{)} \PY{k}{for} \PY{n}{output} \PY{o+ow}{in} \PY{n}{outputs}\PY{p}{]}\PY{p}{)}\PY{p}{)}
\PY{n}{solver}\PY{o}{.}\PY{n}{add}\PY{p}{(}\PY{n}{Or}\PY{p}{(}\PY{p}{[}\PY{n}{output} \PY{o}{==} \PY{n}{BitVecVal}\PY{p}{(}\PY{l+m+mi}{1}\PY{p}{,} \PY{l+m+mi}{1}\PY{p}{)} \PY{k}{for} \PY{n}{output} \PY{o+ow}{in} \PY{n}{outputs}\PY{p}{]}\PY{p}{)}\PY{p}{)}

\PY{c+c1}{\PYZsh{} Verificar se o \PYZdq{}burst\PYZdq{} de zeros é atingível}
\PY{k}{if} \PY{n}{solver}\PY{o}{.}\PY{n}{check}\PY{p}{(}\PY{p}{)} \PY{o}{==} \PY{n}{sat}\PY{p}{:}
    \PY{n+nb}{print}\PY{p}{(}\PY{l+s+s2}{\PYZdq{}}\PY{l+s+s2}{Foi encontrado um burst de zeros dentro do limite.}\PY{l+s+s2}{\PYZdq{}}\PY{p}{)}
\PY{k}{else}\PY{p}{:}
    \PY{n+nb}{print}\PY{p}{(}\PY{l+s+s2}{\PYZdq{}}\PY{l+s+s2}{Não foi encontrado nenhum burst de zeros dentro do limite.}\PY{l+s+s2}{\PYZdq{}}\PY{p}{)}
\end{Verbatim}
\end{tcolorbox}

    \begin{Verbatim}[commandchars=\\\{\}]
Não foi encontrado nenhum burst de zeros dentro do limite.
    \end{Verbatim}

    \hypertarget{ii-ocorruxeancia-de-um-burst-de-tamanho-t-que-repete-um-burst-anterior-no-mesmo-output-em-2t2-passos-ou-menos.}{%
\subsubsection{ii) Ocorrência de um ``burst'' de tamanho t que repete um
``burst'' anterior no mesmo output em 2\^{}(t/2) passos ou
menos.}\label{ii-ocorruxeancia-de-um-burst-de-tamanho-t-que-repete-um-burst-anterior-no-mesmo-output-em-2t2-passos-ou-menos.}}

Tente codificar estas propriedade e cerificar se são acessíveis a partir
de um estado inicial aleatoriamente gerado

    \begin{tcolorbox}[breakable, size=fbox, boxrule=1pt, pad at break*=1mm,colback=cellbackground, colframe=cellborder]
\prompt{In}{incolor}{192}{\boxspacing}
\begin{Verbatim}[commandchars=\\\{\}]
\PY{k+kn}{from} \PY{n+nn}{z3} \PY{k+kn}{import} \PY{o}{*}
\PY{k+kn}{from} \PY{n+nn}{random} \PY{k+kn}{import} \PY{n}{getrandbits}

\PY{c+c1}{\PYZsh{} Definir os parâmetros da propriedade de erro}
\PY{n}{t} \PY{o}{=} \PY{l+m+mi}{10}  \PY{c+c1}{\PYZsh{} Tamanho do \PYZdq{}burst\PYZdq{}}
\PY{n}{limite\PYZus{}passos} \PY{o}{=} \PY{l+m+mi}{2} \PY{o}{*}\PY{o}{*} \PY{p}{(}\PY{n}{t} \PY{o}{/}\PY{o}{/} \PY{l+m+mi}{2}\PY{p}{)}  \PY{c+c1}{\PYZsh{} Número máximo de passos permitidos entre repetições do \PYZdq{}burst\PYZdq{}}

\PY{c+c1}{\PYZsh{} Configurar o solver}
\PY{n}{solver} \PY{o}{=} \PY{n}{Solver}\PY{p}{(}\PY{p}{)}

\PY{c+c1}{\PYZsh{} Inicializar os LFSRs com estados aleatórios}
\PY{n}{curLFSR1} \PY{o}{=} \PY{n}{BitVecVal}\PY{p}{(}\PY{n}{getrandbits}\PY{p}{(}\PY{l+m+mi}{19}\PY{p}{)}\PY{p}{,} \PY{l+m+mi}{19}\PY{p}{)}
\PY{n}{curLFSR2} \PY{o}{=} \PY{n}{BitVecVal}\PY{p}{(}\PY{n}{getrandbits}\PY{p}{(}\PY{l+m+mi}{22}\PY{p}{)}\PY{p}{,} \PY{l+m+mi}{22}\PY{p}{)}
\PY{n}{curLFSR3} \PY{o}{=} \PY{n}{BitVecVal}\PY{p}{(}\PY{n}{getrandbits}\PY{p}{(}\PY{l+m+mi}{23}\PY{p}{)}\PY{p}{,} \PY{l+m+mi}{23}\PY{p}{)}

\PY{n}{output\PYZus{}history} \PY{o}{=} \PY{p}{[}\PY{p}{]}
\PY{n}{found\PYZus{}repetition} \PY{o}{=} \PY{k+kc}{False}

\PY{c+c1}{\PYZsh{} Loop para simular os passos e procurar repetição de um \PYZdq{}burst\PYZdq{}}
\PY{k}{for} \PY{n}{passo} \PY{o+ow}{in} \PY{n+nb}{range}\PY{p}{(}\PY{n}{limite\PYZus{}passos}\PY{p}{)}\PY{p}{:}
    \PY{c+c1}{\PYZsh{} Calcular o bit majoritário}
    \PY{n}{majority\PYZus{}bit} \PY{o}{=} \PY{n}{majority}\PY{p}{(}\PY{n}{Extract}\PY{p}{(}\PY{l+m+mi}{8}\PY{p}{,} \PY{l+m+mi}{8}\PY{p}{,} \PY{n}{curLFSR1}\PY{p}{)}\PY{p}{,} \PY{n}{Extract}\PY{p}{(}\PY{l+m+mi}{10}\PY{p}{,} \PY{l+m+mi}{10}\PY{p}{,} \PY{n}{curLFSR2}\PY{p}{)}\PY{p}{,} \PY{n}{Extract}\PY{p}{(}\PY{l+m+mi}{10}\PY{p}{,} \PY{l+m+mi}{10}\PY{p}{,} \PY{n}{curLFSR3}\PY{p}{)}\PY{p}{)}
    
    \PY{c+c1}{\PYZsh{} Atualizar os LFSRs com base no bit de clock majoritário}
    \PY{n}{next\PYZus{}LFSR1} \PY{o}{=} \PY{n}{If}\PY{p}{(}\PY{n}{Extract}\PY{p}{(}\PY{l+m+mi}{8}\PY{p}{,} \PY{l+m+mi}{8}\PY{p}{,} \PY{n}{curLFSR1}\PY{p}{)} \PY{o}{==} \PY{n}{majority\PYZus{}bit}\PY{p}{,} \PY{n}{lfsr1\PYZus{}seguinte}\PY{p}{(}\PY{n}{curLFSR1}\PY{p}{)}\PY{p}{,} \PY{n}{curLFSR1}\PY{p}{)}
    \PY{n}{next\PYZus{}LFSR2} \PY{o}{=} \PY{n}{If}\PY{p}{(}\PY{n}{Extract}\PY{p}{(}\PY{l+m+mi}{10}\PY{p}{,} \PY{l+m+mi}{10}\PY{p}{,} \PY{n}{curLFSR2}\PY{p}{)} \PY{o}{==} \PY{n}{majority\PYZus{}bit}\PY{p}{,} \PY{n}{lfsr2\PYZus{}seguinte}\PY{p}{(}\PY{n}{curLFSR2}\PY{p}{)}\PY{p}{,} \PY{n}{curLFSR2}\PY{p}{)}
    \PY{n}{next\PYZus{}LFSR3} \PY{o}{=} \PY{n}{If}\PY{p}{(}\PY{n}{Extract}\PY{p}{(}\PY{l+m+mi}{10}\PY{p}{,} \PY{l+m+mi}{10}\PY{p}{,} \PY{n}{curLFSR3}\PY{p}{)} \PY{o}{==} \PY{n}{majority\PYZus{}bit}\PY{p}{,} \PY{n}{lfsr3\PYZus{}seguinte}\PY{p}{(}\PY{n}{curLFSR3}\PY{p}{)}\PY{p}{,} \PY{n}{curLFSR3}\PY{p}{)}
    
    \PY{c+c1}{\PYZsh{} Atualizar o estado atual dos LFSRs}
    \PY{n}{curLFSR1}\PY{p}{,} \PY{n}{curLFSR2}\PY{p}{,} \PY{n}{curLFSR3} \PY{o}{=} \PY{n}{next\PYZus{}LFSR1}\PY{p}{,} \PY{n}{next\PYZus{}LFSR2}\PY{p}{,} \PY{n}{next\PYZus{}LFSR3}
    
    \PY{c+c1}{\PYZsh{} Extrair o bit menos significativo para formar a saída}
    \PY{n}{output\PYZus{}bit} \PY{o}{=} \PY{n}{Extract}\PY{p}{(}\PY{l+m+mi}{0}\PY{p}{,} \PY{l+m+mi}{0}\PY{p}{,} \PY{n}{curLFSR1}\PY{p}{)} \PY{o}{\PYZca{}} \PY{n}{Extract}\PY{p}{(}\PY{l+m+mi}{0}\PY{p}{,} \PY{l+m+mi}{0}\PY{p}{,} \PY{n}{curLFSR2}\PY{p}{)} \PY{o}{\PYZca{}} \PY{n}{Extract}\PY{p}{(}\PY{l+m+mi}{0}\PY{p}{,} \PY{l+m+mi}{0}\PY{p}{,} \PY{n}{curLFSR3}\PY{p}{)}
    
    \PY{c+c1}{\PYZsh{} Adicionar o bit ao histórico de saída}
    \PY{n}{output\PYZus{}history}\PY{o}{.}\PY{n}{append}\PY{p}{(}\PY{n}{output\PYZus{}bit}\PY{p}{)}                           
    
    \PY{c+c1}{\PYZsh{} Verificar se uma sequência de comprimento `t` se repete no histórico}
    \PY{k}{if} \PY{n+nb}{len}\PY{p}{(}\PY{n}{output\PYZus{}history}\PY{p}{)} \PY{o}{\PYZgt{}}\PY{o}{=} \PY{n}{t} \PY{o}{*} \PY{l+m+mi}{2}\PY{p}{:} \PY{c+c1}{\PYZsh{} Garantir que há pelo menos duas sequências para comparar}
        \PY{n}{bits} \PY{o}{=} \PY{p}{[}\PY{p}{]}
        \PY{k}{for} \PY{n}{bit} \PY{o+ow}{in} \PY{n}{output\PYZus{}history}\PY{p}{:}
            \PY{n}{bits}\PY{o}{.}\PY{n}{append}\PY{p}{(}\PY{n}{simplify}\PY{p}{(}\PY{n}{bit}\PY{p}{)}\PY{p}{)} \PY{c+c1}{\PYZsh{} retirar o bit da representação simbólica do Extract}
        \PY{c+c1}{\PYZsh{} Comparar a sequência atual de comprimento `t` com as anteriores}
        \PY{n}{burst\PYZus{}sequence} \PY{o}{=} \PY{n}{bits}\PY{p}{[}\PY{o}{\PYZhy{}}\PY{n}{t}\PY{p}{:}\PY{p}{]}  \PY{c+c1}{\PYZsh{} Últimos `t` bits}
        \PY{k}{for} \PY{n}{i} \PY{o+ow}{in} \PY{n+nb}{range}\PY{p}{(}\PY{n+nb}{len}\PY{p}{(}\PY{n}{bits}\PY{p}{)} \PY{o}{\PYZhy{}} \PY{n}{t}\PY{o}{*}\PY{l+m+mi}{2} \PY{o}{+} \PY{l+m+mi}{1}\PY{p}{)}\PY{p}{:}
            \PY{k}{if} \PY{n}{burst\PYZus{}sequence} \PY{o}{==} \PY{n}{bits}\PY{p}{[}\PY{n}{i}\PY{p}{:}\PY{n}{i} \PY{o}{+} \PY{n}{t}\PY{p}{]}\PY{p}{:}
                \PY{c+c1}{\PYZsh{} Encontramos uma repetição do burst}
                \PY{n+nb}{print}\PY{p}{(}\PY{n}{burst\PYZus{}sequence}\PY{p}{)}
                \PY{n}{found\PYZus{}repetition} \PY{o}{=} \PY{k+kc}{True}
                \PY{n}{solver}\PY{o}{.}\PY{n}{add}\PY{p}{(}\PY{n}{output\PYZus{}bit} \PY{o}{==} \PY{n}{burst\PYZus{}sequence}\PY{p}{[}\PY{l+m+mi}{0}\PY{p}{]}\PY{p}{)}  \PY{c+c1}{\PYZsh{} Exemplo de condição de restrição}
                \PY{k}{break}
    \PY{k}{if} \PY{n}{found\PYZus{}repetition}\PY{p}{:}
        \PY{k}{break}

\PY{c+c1}{\PYZsh{} Verificar se o solver encontrou uma solução para a repetição}
\PY{k}{if} \PY{n}{solver}\PY{o}{.}\PY{n}{check}\PY{p}{(}\PY{p}{)} \PY{o}{==} \PY{n}{sat} \PY{o+ow}{and} \PY{n}{found\PYZus{}repetition}\PY{p}{:}
    \PY{n+nb}{print}\PY{p}{(}\PY{l+s+s2}{\PYZdq{}}\PY{l+s+s2}{Repetição de burst encontrada dentro do limite de passos.}\PY{l+s+s2}{\PYZdq{}}\PY{p}{)}
\PY{k}{else}\PY{p}{:}
    \PY{n+nb}{print}\PY{p}{(}\PY{l+s+s2}{\PYZdq{}}\PY{l+s+s2}{Não foi encontrada nenhuma repetição de burst dentro do limite de passos.}\PY{l+s+s2}{\PYZdq{}}\PY{p}{)}
\end{Verbatim}
\end{tcolorbox}

    \begin{Verbatim}[commandchars=\\\{\}]
[1, 1, 1, 1, 1, 1, 1, 1, 1, 1]
Repetição de burst encontrada dentro do limite de passos.
    \end{Verbatim}


    % Add a bibliography block to the postdoc
    
    
    
\end{document}
